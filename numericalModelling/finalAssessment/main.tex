\documentclass[12pt,a4]{article}
\usepackage{physics, amsmath,amsfonts,amsthm,amssymb, mathtools,steinmetz, gensymb, siunitx}	% LOADS USEFUL MATH STUFF
\usepackage{xcolor,graphicx}
\usepackage[left=45pt, top=60pt, right=45pt, bottom=65pt ,a4paper]{geometry} 				% ADJUSTS PAGE
\usepackage{setspace}
\usepackage{caption}
\usepackage{tikz}
\usepackage{pgf,tikz,pgfplots,wrapfig}
\usepackage{mathrsfs}
\usepackage{fancyhdr}
\usepackage{float}
\usepackage{array}
\usepackage{booktabs,multirow}
\usepackage{bm}

\usetikzlibrary{decorations.text, calc}
\pgfplotsset{compat=1.7}

\usetikzlibrary{decorations.pathreplacing,decorations.markings}
\usepgfplotslibrary{fillbetween}

\newcommand{\vect}[1]{\boldsymbol{#1}}

\usepackage{hyperref}
%\usepackage[style= ACM-Reference-Format, maxbibnames=6, minnames=1,maxnames = 1]{biblatex}
%\addbibresource{references.bib}


\AtBeginDocument{\hypersetup{pdfborder={0 0 0}}}

\title{
\textsc{Numerical Modelling final assessment}
}
\author{\textsc{J L Gouws}
}
\date{\today
\\[1cm]}



\usepackage{graphicx}
\usepackage{array}




\begin{document}
\thispagestyle{empty}

\maketitle

\begin{enumerate}
  \item
  \item
    \begin{enumerate}
      \item
        \begin{align*}
                      & f(x + h) = f(x_0) + h \frac{\partial f(x_0)}{\partial x} + \frac{h^2}{2!} \frac{\partial^2 f(x_0)}{\partial x^2} + \frac{h^3}{3!} \frac{\partial^3 f(x_0)}{\partial x^3} + \frac{h^4}{4!} \frac{\partial^4 f(x_0)}{\partial x^4} + \mathscr{O}(h^5)\\ %+ \frac{h^5}{5!} \frac{\partial^5 f(\xi ')}{\partial x^5}\\
          \Rightarrow & \frac{h^3}{3!} \frac{\partial^3 f(x_0)}{\partial x^3} = f(x + h) - f(x_0) - h \frac{\partial f(x_0)}{\partial x} - \frac{h^2}{2!} \frac{\partial^2 f(x_0)}{\partial x^2} - \frac{h^4}{4!} \frac{\partial^4 f(x_0)}{\partial x^4} + \mathscr{O}(h^5) %- \frac{h^5}{5!} \frac{\partial^5 f(\xi')}{\partial x^5}
        \end{align*}
        We also have
        \begin{align*}
                      & f(x - h) = f(x_0) - h \frac{\partial f(x_0)}{\partial x} + \frac{h^2}{2!} \frac{\partial^2 f(x_0)}{\partial x^2} - \frac{h^3}{3!} \frac{\partial^3 f(x_0)}{\partial x^3} + \frac{h^4}{4!} \frac{\partial^4 f(x_0)}{\partial x^4} + \mathscr{O}(h^5)\\ %- \frac{h^5}{5!} \frac{\partial^5 f(\xi'')}{\partial x^5}\\
          \Rightarrow & \frac{h^3}{3!} \frac{\partial^3 f(x_0)}{\partial x^3} = f(x_0) - h \frac{\partial f(x_0)}{\partial x} + \frac{h^2}{2!} \frac{\partial^2 f(x_0)}{\partial x^2} - f(x - h) + \frac{h^4}{4!} \frac{\partial^4 f(x_0)}{\partial x^4} + \mathscr{O}(h^5) %- \frac{h^5}{5!} \frac{\partial^5 f(\xi'')}{\partial x^5}
        \end{align*}
        Adding these expressions together, we get:
        \begin{align}
          & \frac{h^3}{3} \frac{\partial^3 f(x_0)}{\partial x^3} = f(x + h) - f(x - h) - 2 h \frac{\partial f(x_0)}{\partial x} + \mathscr{O}(h^5) \label{eq:thirdDevFirst}%\frac{h^5}{5!} \frac{\partial^5 f(\mu')}{\partial x^5} \label{eq:thirdDevFirst}
        \end{align}
        From this it can be seen that this estimate is at best $\mathscr{O}(h^2)$.
        The estimate of $\frac{\partial f(x_0)}{\partial x}$ must be $\mathscr{O}(h^4)$ or better for the third dertivative estimate to retain its accuracy.
        Note:
        \begin{align*}
          f(x + h) = f(x_0) + h \frac{\partial f(x_0)}{\partial x} + \frac{h^2}{2} \frac{\partial^2 f(x_0)}{\partial x^2} + \frac{h^3}{6} \frac{\partial^3 f(x_0)}{\partial x^3} + \frac{h^4}{24} \frac{\partial^4 f(x_0)}{\partial x^4} + \mathscr{O}(h^5) %+ \frac{h^4}{24} \frac{\partial^4 f(\chi_1)}{\partial x^4}
        \end{align*}
        \begin{align*}
          f(x - h) = f(x_0) - h \frac{\partial f(x_0)}{\partial x} + \frac{h^2}{2} \frac{\partial^2 f(x_0)}{\partial x^2} - \frac{h^3}{6} \frac{\partial^3 f(x_0)}{\partial x^3} + \frac{h^4}{24} \frac{\partial^4 f(x_0)}{\partial x^4} + \mathscr{O}(h^5) %+ \frac{h^4}{24} \frac{\partial^4 f(\chi_2)}{\partial x^4}
        \end{align*}
        \begin{align*}
          f(x + 2 h) = f(x_0) + 2 h \frac{\partial f(x_0)}{\partial x} + 2 h^2 \frac{\partial^2 f(x_0)}{\partial x^2} + \frac{4 h^3}{3} \frac{\partial^3 f(x_0)}{\partial x^3} + \frac{2 h^4}{3} \frac{\partial^4 f(x_0)}{\partial x^4} + \mathscr{O}(h^5) %+ \frac{2 h^4}{3} \frac{\partial^4 f(\chi_3)}{\partial x^4}
        \end{align*}
        \begin{align*}
          f(x - 2 h) = f(x_0) - 2 h \frac{\partial f(x_0)}{\partial x} + 2 h^2 \frac{\partial^2 f(x_0)}{\partial x^2} - \frac{4 h^3}{3} \frac{\partial^3 f(x_0)}{\partial x^3} + \frac{2 h^4}{3} \frac{\partial^4 f(x_0)}{\partial x^4} + \mathscr{O}(h^5) %+ \frac{2 h^4}{3} \frac{\partial^4 f(\chi_4)}{\partial x^4}
        \end{align*}
        Manipulation of these terms results in the cancellation of unwanted terms:
        \begin{align*}
          f(x + h) - f(x - h) =  2 h \frac{\partial f(x_0)}{\partial x} + \frac{h^3}{3} \frac{\partial^3 f(x_0)}{\partial x^3} + \mathscr{O}(h^5) %+ \frac{h^4}{24} \frac{\partial^4 f(\xi_1)}{\partial x^4}
        \end{align*}
        \begin{align*}
          f(x + 2 h) - f(x - 2 h) = 4 h \frac{\partial f(x_0)}{\partial x} + \frac{8 h^3}{3} \frac{\partial^3 f(x_0)}{\partial x^3} + \mathscr{O}(h^5) %+ \frac{2 h^4}{3} \frac{\partial^4 f(\xi_2)}{\partial x^4}
        \end{align*}
        And finally:
        \begin{align*}
          f(x - 2 h)- 8 f(x - h) + 8 f(x + h) - f(x + 2 h) =  12 h \frac{\partial f(x_0)}{\partial x} + \mathscr{O}(h^4) %+ \frac{h^4}{24} \frac{\partial^4 f(\xi_3)}{\partial x^4}
        \end{align*}
        From this we can get a second order approximation of $f$'s first deriviative:
        \begin{align*}
          \frac{\partial f(x_0)}{\partial x} = \frac{f(x - 2 h)- 8 f(x - h) + 8 f(x + h) - f(x - 2 h)}{12 h} + \mathscr{O}(h^4) %- \frac{h^4}{24} \frac{\partial^4 f(\xi_3)}{\partial x^4}
        \end{align*}
        This expression in combination with Eq.~\ref{eq:thirdDevFirst} gives the second order approximation of $u_{xxx}$:
        \begin{align*}
                      & \frac{h^3}{3} \frac{\partial^3 f(x_0)}{\partial x^3} = f(x + h) - f(x - h) - \frac{f(x - 2 h)- 8 f(x - h) + 8 f(x + h) - f(x + 2 h)}{6} + \mathscr{O}(h^5)\\ %+ \frac{h^5}{5!} \frac{\partial^5 f(\mu')}{\partial x^5}\\
          \Rightarrow & \frac{\partial^3 f(x_0)}{\partial x^3} = \frac{3 f(x + h) - 3 f(x - h)}{h^3} - \frac{f(x - 2 h)- 8 f(x - h) + 8 f(x + h) - f(x + 2 h)}{2 h^3} + \mathscr{O}(h^2)\\ %+ \frac{h^5}{5!} \frac{\partial^5 f(\mu')}{\partial x^5}\\
          \Rightarrow & \frac{\partial^3 f(x_0)}{\partial x^3} = \frac{- f(x - 2 h) + 2 f(x - h) - 2 f(x + h) + f(x +  2 h)}{2 h^3} + \mathscr{O}(h^2) %+ \frac{h^5}{5!} \frac{\partial^5 f(\mu')}{\partial x^5} 
        \end{align*}
      \item
        \begin{enumerate}
          \item
            Figure~\ref{fig:motion1} shows a collision of a soliton wave.
            The soliton wave solves the KdV equation with the initial form:
            \begin{equation*}
              u(x, 0) = 2 \sech ^ 2 (x - 4t)
            \end{equation*}
            \begin{figure}[H]
              \centering
              \includegraphics[scale = 0.8]{figs/motion1.pdf}
              \caption{Motion of soliton wave}
              \label{fig:motion1}
            \end{figure}
          \item
            Figure~\ref{fig:KDVmass1} shows the mass of the wave as a function of time.
            \begin{figure}[H]
              \centering
              \includegraphics[scale = 0.8]{figs/KDVmass1.pdf}
              \caption{The mass for the wave}
              \label{fig:KDVmass1}
            \end{figure}
        \end{enumerate}
      \item
        \begin{enumerate}
          \item
            Figure~\ref{fig:collision} shows a collision of a soliton wave.
            The soliton wave solves the KdV equation with the initial form:
            \begin{equation*}
              u(x, 0) = \frac{12[3 + 4 \cosh(2 x) + \cosh(4x)]}{[3 \cosh(x) + \cosh(3x)]^2}
            \end{equation*}
            \begin{figure}[H]
              \centering
              \includegraphics[scale = 0.8]{figs/collision.pdf}
              \caption{Collision of soliton wave}
              \label{fig:collision}
            \end{figure}
          \item
            Figure~\ref{fig:KDVmass2} shows the mass of the wave as a function of time.
            The mass is not an absolute constant with time.
            There are two bumps in the mass, which occur at the beginning and end of the collision.
            \begin{figure}[H]
              \centering
              \includegraphics[scale = 0.8]{figs/KDVmass2.pdf}
              \caption{The mass for the wave}
              \label{fig:KDVmass2}
            \end{figure}
        \end{enumerate}
    \end{enumerate}
\end{enumerate}

\end{document}
