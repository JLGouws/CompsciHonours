\documentclass[12pt,a4]{article}
\usepackage{physics, amsmath,amsfonts,amsthm,amssymb, mathtools,steinmetz, gensymb, siunitx}	% LOADS USEFUL MATH STUFF
\usepackage{xcolor,graphicx}
\usepackage[left=45pt, top=60pt, right=45pt, bottom=65pt ,a4paper]{geometry} 				% ADJUSTS PAGE
\usepackage{setspace}
\usepackage{caption}
\usepackage{tikz}
\usepackage{pgf,tikz,pgfplots,wrapfig}
\usepackage{mathrsfs}
\usepackage{fancyhdr}
\usepackage{float}
\usepackage{array}
\usepackage{booktabs,multirow}
\usepackage{bm}

\usetikzlibrary{decorations.text, calc}
\pgfplotsset{compat=1.7}

\usetikzlibrary{decorations.pathreplacing,decorations.markings}
\usepgfplotslibrary{fillbetween}

\newcommand{\vect}[1]{\boldsymbol{#1}}

\usepackage{hyperref}
%\usepackage[style= ACM-Reference-Format, maxbibnames=6, minnames=1,maxnames = 1]{biblatex}
%\addbibresource{references.bib}


\AtBeginDocument{\hypersetup{pdfborder={0 0 0}}}

\title{
\textsc{Numerical Modelling final assessment}
}
\author{\textsc{J L Gouws}
}
\date{\today
\\[1cm]}



\usepackage{graphicx}
\usepackage{array}



\begin{document}
\thispagestyle{empty}

\maketitle

\begin{enumerate}
  \item
    \begin{enumerate}
      \item
        A possible implementation of Neumann boundary conditions involves predicting a ghost point using the boundary condition.
        An example of this at the left boundary, $\nu_0^n$ is:
        \begin{align*}
                      & {\nu_x}_0^j = \frac{\nu_1^j - \nu_{-1}^j}{2 h} + \mathscr{O} (h^2)\\
          \Rightarrow & v_{-1}^j = 2 h {\nu_x}_0^j + \nu_{-1}^j + \mathscr{O} (h^3)
        \end{align*}
        The calculation of the second derivative at the boundary can use this ghost point.
        With this choice, the second derivative at the boundaries will only be correct to first order.

        \iffalse
        Forward and backward finite difference methods can give a second order accurate second derivative at the boundary.
        For the left most point:
        \begin{equation*}
          {\nu_{x}}_1^j = \frac{- 25 \nu_1^j / 12 + 4 \nu_2^j - 3 \nu_3^j  + 4 \nu_4^j / 3 - \nu_5^j / 4}{h} + \mathscr {O}(h^4)
        \end{equation*}
        and:
        \begin{equation*}
          {\nu_{x}}_2^j = \frac{\nu_0^j / 12 - 2 \nu_1^j / 3 + 2 \nu_3^j / 3 - \nu_4^j / 12}{h} + \mathscr {O}(h^4)
        \end{equation*}
        These can be combined to form a second order approximation of the second derivative:
        \begin{equation*}
          {\nu_{xx}}_0^j = \frac{- 3 h {\nu_x}_0^j + \left(-\nu_0^j / 12 - 9 \nu_1^j + 16 \nu_2^j - 38 \nu_3^j / 3 + 49 \nu_4^j/ 12 - v_5^j\right)}{2 h^2} + \mathscr {O}(h^2)
        \end{equation*}
        For the right most end point:
        \begin{equation*}
          {\nu_{x}}_{N-1}^j = \frac{- \nu_{N-4}^j / 3 + 3 \nu_{N-3}^j / 2 - 3 \nu_{N-2}^j + 11 \nu_{N-1}^j / 6}{h} + \mathscr {O}(h^3)
        \end{equation*}
        and:
        \begin{equation*}
          {\nu_{x}}_{N-2}^j = \frac{\nu_{N-4}^j / 12 - 2 \nu_{N-3}^j / 3 + 2 \nu_{N-1}^j / 3 - \nu_N^j / 12}{h} + \mathscr {O}(h^4)
        \end{equation*}
        These can be combined to form a second order approximation of the second derivative:
        \begin{equation*}
          {\nu_{xx}}_N^j = \frac{h\left(17 \nu_{N-4}^j/ 12 - 20 \nu_{N-3}^j/ 3 + 12 \nu_{N-2}^j + 20 \nu_{N - 1}^j/ 3 - \nu_N^j / 12\right) + 3 {\nu_x}_N^j}{2 h^2} + \mathscr {O}(h^2)
        \end{equation*}
        \fi
        An RK4 integrator evolves the surface in time.
        Figure~\ref{fig:profile} shows the evolution of a heat equation with initial profile:
        \begin{equation*}
          u(x, y ,0) = e^{-[(x- x_0)^2 + (y - y_0)^2]}
        \end{equation*}
        with $\sigma = 0.01$.
        \begin{figure}[H]
          \centering
          \includegraphics[scale = 0.8]{figs/profile.pdf}
          \caption{Profile of a heat equation evolution}
          \label{fig:profile}
        \end{figure}
        Another option of handling the boundary conditions is:
        \begin{equation*}
          f(x_0 + h ) = f(x_0) + h \frac{d f(x_0)}{dx} + \frac{h^2}{2} \frac{d^2 f(x_0)}{dx^2} + \frac{h^3}{6} \frac{d^3 f(x_0)}{dx^3} + \mathscr{O} (h^4)
        \end{equation*}
        And:
        \begin{align*}
                      & f(x_0 + 2h ) = f(x_0) + 2h \frac{d f(x_0)}{dx} + 2h^2 \frac{d^2 f(x_0)}{dx^2} + \frac{4 h^3}{3} \frac{d^3 f(x_0)}{dx^3} + \mathscr{O} (h^4)\\
          \Rightarrow & \frac{h^3}{6} \frac{d^3 f(x_0)}{dx^3} = \frac{f(x_0 + 2h) - f(x_0) }{8} - \frac{h}{4} \frac{d f(x_0)}{dx} - \frac{h^2}{4} \frac{d^2 f(x_0)}{dx^2} + \mathscr{O} (h^4)
        \end{align*}
        Combining these two:
        \begin{align*}
                      & f(x_0 + h) = \frac{7 f(x_0) + f(x_0 + 2h)}{8} + \frac{3 h}{4} \frac{d f(x_0)}{dx} + \frac{h^2}{4} \frac{d^2 f(x_0)}{dx^2} + \mathscr{O} (h^4)\\
          \Rightarrow &  \frac{d^2 f(x_0)}{dx^2} = - \frac{3}{h} \frac{d f(x_0)}{dx} + \frac{- 7 f(x_0) + 8 f(x_0 + h) - f(x_0 + 2h)}{2 h^2} + \mathscr{O} (h^2)
        \end{align*}
        For the right most boundary point:
        \begin{equation*}
          f(x_0 - h ) = f(x_0) - h \frac{d f(x_0)}{dx} + \frac{h^2}{2} \frac{d^2 f(x_0)}{dx^2} - \frac{h^3}{6} \frac{d^3 f(x_0)}{dx^3} + \mathscr{O} (h^4)
        \end{equation*}
        And:
        \begin{align*}
                      & f(x_0 - 2h ) = f(x_0) - 2h \frac{d f(x_0)}{dx} + 2h^2 \frac{d^2 f(x_0)}{dx^2} - \frac{4 h^3}{3} \frac{d^3 f(x_0)}{dx^3} + \mathscr{O} (h^4)\\
          \Rightarrow & \frac{h^3}{6} \frac{d^3 f(x_0)}{dx^3} = \frac{f(x_0) - f(x_0 - 2h)}{8} - \frac{h}{4} \frac{d f(x_0)}{dx} + \frac{h^2}{4} \frac{d^2 f(x_0)}{dx^2} + \mathscr{O} (h^4)
        \end{align*}
        Combining these two:
        \begin{align*}
                      & f(x_0 - h) = \frac{7 f(x_0) + f(x_0 - 2h)}{8} - \frac{3 h}{4} \frac{d f(x_0)}{dx} + \frac{h^2}{4} \frac{d^2 f(x_0)}{dx^2} + \mathscr{O} (h^4)\\
          \Rightarrow &  \frac{d^2 f(x_0)}{dx^2} = \frac{3}{h} \frac{d f(x_0)}{dx} + \frac{- 7 f(x_0) + 8 f(x_0 - h) - f(x_0 - 2h)}{2 h^2} + \mathscr{O} (h^2)
        \end{align*}
        Figure~\ref{fig:secondOrder} shows this second-order method being used to solve a one-dimensional dispersion problem.
        \begin{figure}[H]
          \centering
          \includegraphics[scale = 0.52]{figs/secondOrder.pdf}
          \caption{Second order Neuman end points}
          \label{fig:secondOrder}
        \end{figure}
        
        The rest of this question uses the first-order method described above and an RK4 integrator.
      \item
        Figure~\ref{fig:T1} shows the $T(t)$'s value as a function of time.
        \begin{figure}[H]
          \centering
          \includegraphics[scale = 0.8]{figs/T1.pdf}
          \caption{$T(t)$ for the given 2 dimensional heat equation}
          \label{fig:T1}
        \end{figure}

      \item
        Figure~\ref{fig:T} gives a semi-logarithmic plot of $T(t)$ as a function of time for different resolutions.
        \begin{figure}[H]
          \centering
          \includegraphics[scale = 0.8]{figs/T.pdf}
          \caption{$T(t)$ at various grid resolutions}
          \label{fig:T}
        \end{figure}
      \item
        Figure~\ref{fig:heatSurface} shows how the initial profile changes as time progresses.
        The dye's density falls below $10^{-4}$ parts per area when $t = 0.01524$.

        \begin{figure}[H]
          \centering
          \includegraphics[scale = 0.6]{figs/heatSurface.pdf}
          \caption{Evolution of the surface as time progresses}
          \label{fig:heatSurface}
        \end{figure}
    \end{enumerate}

  \item
    \begin{enumerate}
      \item
        \begin{align*}
                      & f(x + h) = f(x_0) + h \frac{\partial f(x_0)}{\partial x} + \frac{h^2}{2!} \frac{\partial^2 f(x_0)}{\partial x^2} + \frac{h^3}{3!} \frac{\partial^3 f(x_0)}{\partial x^3} + \frac{h^4}{4!} \frac{\partial^4 f(x_0)}{\partial x^4} + \mathscr{O}(h^5)\\ %+ \frac{h^5}{5!} \frac{\partial^5 f(\xi ')}{\partial x^5}\\
          \Rightarrow & \frac{h^3}{3!} \frac{\partial^3 f(x_0)}{\partial x^3} = f(x + h) - f(x_0) - h \frac{\partial f(x_0)}{\partial x} - \frac{h^2}{2!} \frac{\partial^2 f(x_0)}{\partial x^2} - \frac{h^4}{4!} \frac{\partial^4 f(x_0)}{\partial x^4} + \mathscr{O}(h^5) %- \frac{h^5}{5!} \frac{\partial^5 f(\xi')}{\partial x^5}
        \end{align*}
        We also have
        \begin{align*}
                      & f(x - h) = f(x_0) - h \frac{\partial f(x_0)}{\partial x} + \frac{h^2}{2!} \frac{\partial^2 f(x_0)}{\partial x^2} - \frac{h^3}{3!} \frac{\partial^3 f(x_0)}{\partial x^3} + \frac{h^4}{4!} \frac{\partial^4 f(x_0)}{\partial x^4} + \mathscr{O}(h^5)\\ %- \frac{h^5}{5!} \frac{\partial^5 f(\xi'')}{\partial x^5}\\
          \Rightarrow & \frac{h^3}{3!} \frac{\partial^3 f(x_0)}{\partial x^3} = f(x_0) - h \frac{\partial f(x_0)}{\partial x} + \frac{h^2}{2!} \frac{\partial^2 f(x_0)}{\partial x^2} - f(x - h) + \frac{h^4}{4!} \frac{\partial^4 f(x_0)}{\partial x^4} + \mathscr{O}(h^5) %- \frac{h^5}{5!} \frac{\partial^5 f(\xi'')}{\partial x^5}
        \end{align*}
        Adding these expressions together, we get:
        \begin{align}
          & \frac{h^3}{3} \frac{\partial^3 f(x_0)}{\partial x^3} = f(x + h) - f(x - h) - 2 h \frac{\partial f(x_0)}{\partial x} + \mathscr{O}(h^5) \label{eq:thirdDevFirst}%\frac{h^5}{5!} \frac{\partial^5 f(\mu')}{\partial x^5} \label{eq:thirdDevFirst}
        \end{align}
        From this it can be seen that this estimate is at best $\mathscr{O}(h^2)$.
        The estimate of $\frac{\partial f(x_0)}{\partial x}$ must be $\mathscr{O}(h^4)$ or better for the third dertivative estimate to retain its accuracy.
        Note:
        \begin{align*}
          f(x + h) = f(x_0) + h \frac{\partial f(x_0)}{\partial x} + \frac{h^2}{2} \frac{\partial^2 f(x_0)}{\partial x^2} + \frac{h^3}{6} \frac{\partial^3 f(x_0)}{\partial x^3} + \frac{h^4}{24} \frac{\partial^4 f(x_0)}{\partial x^4} + \mathscr{O}(h^5) %+ \frac{h^4}{24} \frac{\partial^4 f(\chi_1)}{\partial x^4}
        \end{align*}
        \begin{align*}
          f(x - h) = f(x_0) - h \frac{\partial f(x_0)}{\partial x} + \frac{h^2}{2} \frac{\partial^2 f(x_0)}{\partial x^2} - \frac{h^3}{6} \frac{\partial^3 f(x_0)}{\partial x^3} + \frac{h^4}{24} \frac{\partial^4 f(x_0)}{\partial x^4} + \mathscr{O}(h^5) %+ \frac{h^4}{24} \frac{\partial^4 f(\chi_2)}{\partial x^4}
        \end{align*}
        \begin{align*}
          f(x + 2 h) = f(x_0) + 2 h \frac{\partial f(x_0)}{\partial x} + 2 h^2 \frac{\partial^2 f(x_0)}{\partial x^2} + \frac{4 h^3}{3} \frac{\partial^3 f(x_0)}{\partial x^3} + \frac{2 h^4}{3} \frac{\partial^4 f(x_0)}{\partial x^4} + \mathscr{O}(h^5) %+ \frac{2 h^4}{3} \frac{\partial^4 f(\chi_3)}{\partial x^4}
        \end{align*}
        \begin{align*}
          f(x - 2 h) = f(x_0) - 2 h \frac{\partial f(x_0)}{\partial x} + 2 h^2 \frac{\partial^2 f(x_0)}{\partial x^2} - \frac{4 h^3}{3} \frac{\partial^3 f(x_0)}{\partial x^3} + \frac{2 h^4}{3} \frac{\partial^4 f(x_0)}{\partial x^4} + \mathscr{O}(h^5) %+ \frac{2 h^4}{3} \frac{\partial^4 f(\chi_4)}{\partial x^4}
        \end{align*}
        Manipulation of these terms results in the cancellation of unwanted terms:
        \begin{align*}
          f(x + h) - f(x - h) =  2 h \frac{\partial f(x_0)}{\partial x} + \frac{h^3}{3} \frac{\partial^3 f(x_0)}{\partial x^3} + \mathscr{O}(h^5) %+ \frac{h^4}{24} \frac{\partial^4 f(\xi_1)}{\partial x^4}
        \end{align*}
        \begin{align*}
          f(x + 2 h) - f(x - 2 h) = 4 h \frac{\partial f(x_0)}{\partial x} + \frac{8 h^3}{3} \frac{\partial^3 f(x_0)}{\partial x^3} + \mathscr{O}(h^5) %+ \frac{2 h^4}{3} \frac{\partial^4 f(\xi_2)}{\partial x^4}
        \end{align*}
        And finally:
        \begin{align*}
          f(x - 2 h)- 8 f(x - h) + 8 f(x + h) - f(x + 2 h) =  12 h \frac{\partial f(x_0)}{\partial x} + \mathscr{O}(h^4) %+ \frac{h^4}{24} \frac{\partial^4 f(\xi_3)}{\partial x^4}
        \end{align*}
        From this we can get a second order approximation of $f$'s first deriviative:
        \begin{align*}
          \frac{\partial f(x_0)}{\partial x} = \frac{f(x - 2 h)- 8 f(x - h) + 8 f(x + h) - f(x - 2 h)}{12 h} + \mathscr{O}(h^4) %- \frac{h^4}{24} \frac{\partial^4 f(\xi_3)}{\partial x^4}
        \end{align*}
        This expression in combination with Eq.~\ref{eq:thirdDevFirst} gives the second order approximation of $u_{xxx}$:
        \begin{align*}
                      & \frac{h^3}{3} \frac{\partial^3 f(x_0)}{\partial x^3} = f(x + h) - f(x - h) - \\
                      & \qquad \qquad \qquad \qquad \frac{f(x - 2 h)- 8 f(x - h) + 8 f(x + h) - f(x + 2 h)}{6} + \mathscr{O}(h^5)\\ %+ \frac{h^5}{5!} \frac{\partial^5 f(\mu')}{\partial x^5}\\
%          \Rightarrow & \frac{\partial^3 f(x_0)}{\partial x^3} = \frac{3 f(x + h) - 3 f(x - h)}{h^3} - \frac{f(x - 2 h)- 8 f(x - h) + 8 f(x + h) - f(x + 2 h)}{2 h^3} + \mathscr{O}(h^2)\\ %+ \frac{h^5}{5!} \frac{\partial^5 f(\mu')}{\partial x^5}\\
          \Rightarrow & \frac{\partial^3 f(x_0)}{\partial x^3} = \frac{- f(x - 2 h) + 2 f(x - h) - 2 f(x + h) + f(x +  2 h)}{2 h^3} + \mathscr{O}(h^2) %+ \frac{h^5}{5!} \frac{\partial^5 f(\mu')}{\partial x^5} 
        \end{align*}
      \item
        \begin{enumerate}
          \item
            Figure~\ref{fig:KDVmass1} shows the mass of the wave as a function of time.
            The wave has the initial profile given in Eq.~\ref{eq:kdvi}.
            \begin{figure}[H]
              \centering
              \includegraphics[scale = 0.8]{figs/KDVmass1.pdf}
              \caption{The mass for the wave}
              \label{fig:KDVmass1}
            \end{figure}
          \item
            Figure~\ref{fig:motion1} shows the motion of a  soliton wave.
            The soliton wave solves the KdV equation with the initial form:
            \begin{equation}
              u(x, 0) = 2 \sech ^ 2 (x - 4t)
              \label{eq:kdvi}
            \end{equation}
            \begin{figure}[H]
              \centering
              \includegraphics[scale = 0.7]{figs/motion1.pdf}
              \caption{Motion of soliton wave}
              \label{fig:motion1}
            \end{figure}
        \end{enumerate}
      \item
        \begin{enumerate}
          \item
            Figure~\ref{fig:collision} shows a collision of a soliton wave.
            The soliton wave solves the KdV equation with the initial form:
            \begin{equation*}
              u(x, 0) = \frac{12[3 + 4 \cosh(2 x) + \cosh(4x)]}{[3 \cosh(x) + \cosh(3x)]^2}
            \end{equation*}
            \begin{figure}[H]
              \centering
              \includegraphics[scale = 0.8]{figs/collision.pdf}
              \caption{Collision of soliton wave}
              \label{fig:collision}
            \end{figure}
          \item
            Figure~\ref{fig:KDVmass2} shows the mass of the wave as a function of time.
            The mass is not an absolute constant with time.
            There are two bumps in the mass, which occur at the beginning and end of the collision.
            \begin{figure}[H]
              \centering
              \includegraphics[scale = 0.8]{figs/KDVmass2.pdf}
              \caption{The mass for the wave}
              \label{fig:KDVmass2}
            \end{figure}
        \end{enumerate}
    \end{enumerate}
\end{enumerate}

\end{document}
