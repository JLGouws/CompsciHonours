\documentclass[12pt,a4]{article}
\usepackage{physics, amsmath,amsfonts,amsthm,amssymb, mathtools,steinmetz, gensymb, siunitx}	% LOADS USEFUL MATH STUFF
\usepackage{xcolor,graphicx}
\usepackage[left=45pt, top=60pt, right=45pt, bottom=65pt ,a4paper]{geometry} 				% ADJUSTS PAGE
\usepackage{setspace}
\usepackage{caption}
\usepackage{tikz}
\usepackage{pgf,tikz,pgfplots,wrapfig}
\usepackage{mathrsfs}
\usepackage{fancyhdr}
\usepackage{float}
\usepackage{array}
\usepackage{booktabs,multirow}
\usepackage{bm}

\usetikzlibrary{decorations.text, calc}
\pgfplotsset{compat=1.7}

\usetikzlibrary{decorations.pathreplacing,decorations.markings}
\usepgfplotslibrary{fillbetween}

\newcommand{\vect}[1]{\boldsymbol{#1}}

\usepackage{hyperref}
%\usepackage[style= ACM-Reference-Format, maxbibnames=6, minnames=1,maxnames = 1]{biblatex}
%\addbibresource{references.bib}


\AtBeginDocument{\hypersetup{pdfborder={0 0 0}}}

\title{
\textsc{Topic 5}
}
\author{\textsc{J L Gouws}
}
\date{\today
\\[1cm]}



\usepackage{graphicx}
\usepackage{array}




\begin{document}
\thispagestyle{empty}

\maketitle

\begin{enumerate}
  \item
    For the upwind derivative operator:
    \begin{align*}
                  & f(x_0 - h) = f(x_0) - h \frac{\partial f(x_0)}{\partial x} + \frac{h^2}{2!} \frac{\partial^2 f(\mu)}{\partial x^2}\\
      \Rightarrow & \frac{\partial f(x_0)}{\partial x} = \frac{f(x_0) - f(x_0 - h)}{h} - \frac{h}{2} \frac{\partial^2 f(\mu)}{\partial x^2}
    \end{align*}
    For the centered derive operator, taylor expanding the function at the points to the left and right of $x_0$ gives
    \begin{align*}
      f(x_0 + h) = f(x_0) + h \frac{\partial f(x_0)}{\partial x} + \frac{h^2}{2!} \frac{\partial^2 f(x_0)}{\partial x^2} + \frac{h^3}{3!} \frac{\partial^3 f(\xi')}{\partial x^3} 
    \end{align*}
    And:
    \begin{align*}
      f(x_0 - h) = f(x_0) - h \frac{\partial f(x_0)}{\partial x} + \frac{h^2}{2!} \frac{\partial^2 f(x_0)}{\partial x^2} - \frac{h^3}{3!} \frac{\partial^3 f(\xi'')}{\partial x^3}
    \end{align*}
    Subtracting the second from the first:
    \begin{align*}
      & f(x_0 + h) - f(x_0 - h) = 2 h \frac{\partial f(x_0)}{\partial x} + \frac{h^3}{3!} \frac{\partial^3 f(\mu)}{\partial x^3} \\
      \Rightarrow & \frac{\partial f(x_0)}{\partial x} = \frac{f(x_0 + h) - f(x_0 - h)}{ 2 h}  - \frac{h^2}{12} \frac{\partial^3 f(\mu)}{\partial x^3} 
    \end{align*}
    For the second derivative operator, taylor expanding the function at the points to the left and right of $x_0$ gives:
    \begin{align*}
      f(x_0 + h) = f(x_0) + h \frac{\partial f(x_0)}{\partial x} + \frac{h^2}{2!} \frac{\partial^2 f(x_0)}{\partial x^2} + \frac{h^3}{3!} \frac{\partial^3 f(x_0)}{\partial x^3} + \frac{h^4}{4!} \frac{\partial^4 f(\xi')}{\partial x^4}
    \end{align*}
    And:
    \begin{align*}
      f(x_0 - h) = f(x_0) - h \frac{\partial f(x_0)}{\partial x} + \frac{h^2}{2!} \frac{\partial^2 f(x_0)}{\partial x^2} - \frac{h^3}{3!} \frac{\partial^3 f(x_0)}{\partial x^3} + \frac{h^4}{4!} \frac{\partial^4 f(\xi '')}{\partial x^4}
    \end{align*}
    Adding these expressions together, we get:
    \begin{align*}
                  & f(x_0 + h) + f(x_0 - h) =  2 f(x_0) + h^2 \frac{\partial^2 f(x_0)}{\partial x^2} + \frac{h^4}{24} \frac{\partial^4 f(\mu)}{\partial x^4}\\
      \Rightarrow & \frac{\partial^2 f(x_0)}{\partial x^2} =\frac{f(x_0 + h) - 2 f(x_0) + f(x_0 - h)}{h^2}  + \frac{h^2}{24} \frac{\partial^4 f(\mu)}{\partial x^4}
    \end{align*}
  \item
    \begin{enumerate}
      \item
        Figure~\ref{fig:upwindEuler} shows the upwind Euler method evolving an advetion equation.
        The paricular advection problem is given by:
        \begin{equation}
          \begin{gathered}
            u_t + u _x = 0\\
            u(x + 2 \pi) = u(x)\\
            u(x, 0) = \sin(x) \cos^8(x / 2)
          \label{eq:advec}
          \end{gathered}
        \end{equation}
        \begin{figure}[H]
          \centering
          \includegraphics[scale = 0.9]{../figs/upwindEuler.pdf} 
          \caption{Upwind Euler Evolution of an advection equation with $\Delta t = \Delta x$ and error}
          \label{fig:upwindEuler}
        \end{figure}

      \item 
        Figure~\ref{fig:upwindEuler1.1.pdf} shows the upwind Euler solution to the problem in Eq.~\ref{eq:advec} using a timestep that is larger than the grid spacing.
        \begin{figure}[H]
          \centering
          \includegraphics[scale = 0.9]{../figs/upwindEuler1.1.pdf}
          \caption{Upwind Euler Evolution of an advection equation with $\Delta t = 1.1\Delta x$ and error}
          \label{fig:upwindEuler1.1.pdf}
        \end{figure}

      \item
        Figure~\ref{fig:centeredEuler} shows the centered Euler method's solution to Eq.~\ref{eq:advec}'s problem.
        \begin{figure}[H]
          \centering
          \includegraphics[scale = 0.9]{../figs/centeredEuler.pdf}
          \caption{Centered derivative and Euler Evolution of an advection equation  with $\Delta t = \Delta x$ and error}
          \label{fig:centeredEuler}
        \end{figure}
    \end{enumerate}

  \item
    \begin{enumerate}
      \item
        Figure~\ref{fig:heatEulerdx} shows the centered Euler method's solution to Eq.~\ref{eq:heat}'s problem.
        The Euler method uses a timestep equal to the grid spacing.
        \begin{equation}
          \begin{gathered}
            u_t - u_{xx} = 0\\
            - \pi \leq x \leq \pi\\
            u(-\pi, t) = 0 = u(\pi, t)\\
            u(x, 0) = \sin(x) \cos^8(x / 2)
          \end{gathered}
          \label{eq:heat}
        \end{equation}

        \begin{figure}[H]
          \centering
          \includegraphics[scale = 0.8]{../figs/heatEulerdx.pdf}
          \caption{Euler Evolution of a heat equation with $\Delta t = \Delta x$}
          \label{fig:heatEulerdx}
        \end{figure}

      \item
        Figure~\ref{fig:heatEuler01dx} shows the centered Euler method's solution to Eq.~\ref{eq:heat}'s problem using a smaller timestep.
        \begin{figure}[H]
          \centering
          \includegraphics[scale = 0.8]{../figs/heatEuler01dx.pdf}
          \caption{Euler Evolution of a heat equation with $\Delta t = 0.01 \Delta x$}
          \label{fig:heatEuler01dx}
        \end{figure}

      \item
        Figure~\ref{fig:heatEulerMaxdx} shows the maximum timestep for which the centered Euler method is stable.
        An anotation for the graph shows the size of this solution's time step.
        The verge of instability is evident in the last iteration.
        \begin{figure}[H]
          \centering
          \includegraphics[scale = 0.8]{../figs/heatEulerMaxdx.pdf}
          \caption{Maximum $\Delta t$ step size of Euler Evolution of a heat equation}
          \label{fig:heatEulerMaxdx}
        \end{figure}

    \end{enumerate}
  \item
    Figure~\ref{fig:heatRK4Maxdx} shows an RK4 method's solution to the given heat equation.
    An anotation for the graph shows the size of this solution's time step.
    The verge of instability is evident in the last iteration.
    \begin{figure}[H]
      \centering
      \includegraphics[scale = 0.8]{../figs/heatRK4Maxdx.pdf}
      \caption{Maximum $\Delta t$ step size of Runge-Kutta Evolution of a heat equation}
      \label{fig:heatRK4Maxdx}
    \end{figure}
\end{enumerate}

\end{document}
