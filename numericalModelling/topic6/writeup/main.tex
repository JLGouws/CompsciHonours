\documentclass[12pt,a4]{article}
\usepackage{physics, amsmath,amsfonts,amsthm,amssymb, mathtools,steinmetz, gensymb, siunitx}	% LOADS USEFUL MATH STUFF
\usepackage{xcolor,graphicx}
\usepackage[left=45pt, top=60pt, right=45pt, bottom=65pt ,a4paper]{geometry} 				% ADJUSTS PAGE
\usepackage{setspace}
\usepackage{caption}
\usepackage{tikz}
\usepackage{pgf,tikz,pgfplots,wrapfig}
\usepackage{mathrsfs}
\usepackage{fancyhdr}
\usepackage{float}
\usepackage{array}
\usepackage{booktabs,multirow}
\usepackage{bm}

\usetikzlibrary{decorations.text, calc}
\pgfplotsset{compat=1.7}

\usetikzlibrary{decorations.pathreplacing,decorations.markings}
\usepgfplotslibrary{fillbetween}

\newcommand{\vect}[1]{\boldsymbol{#1}}

\usepackage{hyperref}
%\usepackage[style= ACM-Reference-Format, maxbibnames=6, minnames=1,maxnames = 1]{biblatex}
%\addbibresource{references.bib}


\AtBeginDocument{\hypersetup{pdfborder={0 0 0}}}

\title{
\textsc{Topic 6}
}
\author{\textsc{J L Gouws}
}
\date{\today
\\[1cm]}



\usepackage{graphicx}
\usepackage{array}




\begin{document}
\thispagestyle{empty}

\maketitle

\begin{enumerate}
  \item
    \begin{enumerate}
      \item
        Figure~\ref{fig:eulerdx} shows the numerical advection of a top hat function with an Euler integrator.
        For this integration, the evolution time step is equal to the spatial variable's point spacing.
        The figure indicates that the method is neither dissipative nor dispersive.
        \begin{figure}[H]
          \centering
          \includegraphics[scale = 0.85]{../figs/eulerdx.pdf}
          \caption{Euler Evolution of an advection equation with $\Delta t = \Delta x$}
          \label{fig:eulerdx}
        \end{figure}
      \item
        Figure~\ref{fig:euler.8dx} shows the numerical advection of a top hat function with an Euler integrator.
        For this integration, the evolution time step is equal to an eighth of the spatial variable's point spacing.
        The figure indicates that the method is dissipative.
        \begin{figure}[H]
          \centering
          \includegraphics[scale = 0.85]{../figs/euler.8dx.pdf}
          \caption{Euler Evolution of an advection equation with $\Delta t = 0.8 \Delta x$}
          \label{fig:euler.8dx}
        \end{figure}
    \end{enumerate}
  \item
    \begin{enumerate}
      \item
        The error propagation for the Lax-Wendroff method is given by:
        \begin{align*}
                      & u^{n+1}_j - \varepsilon^{n+1}_j = u^{n}_j - \varepsilon^{n}_j - \frac{c \Delta t}{2\Delta x} (u^{n}_{j + 1} - \varepsilon^{n}_{j+1} - u^{n}_{j - 1} + \varepsilon^{n}_{j-1})\\
                      & \qquad \qquad \qquad \qquad + \frac{c^2 \Delta^2}{2 \Delta x^2}(u^n_{j + 1} - \varepsilon^n_{j+1} - 2(u^n_j - \varepsilon^n_j) + u^n_{j-1} - \varepsilon^n_{j-1})\\
          \Rightarrow & \frac{u^{n+1}_j - u^n_j}{\Delta t^2} + c \frac{u^{n}_{j + 1} - u^{n}_{j - 1}}{2 \Delta x \Delta t} - \frac{c^2 }{2 \Delta x^2}(u^n_{j + 1} - 2 u^n_j + u^n_{j-1})\\
                      &  \qquad = \frac{\varepsilon^{n+1}_j - \varepsilon^n_j}{\Delta t^2} + c \frac{\varepsilon^{n}_{j + 1} - \varepsilon^{n}_{j - 1}}{2 \Delta x \Delta t} - \frac{c^2 }{2 \Delta x^2}(\varepsilon^n_{j + 1} - 2 \varepsilon^n_j + \varepsilon^n_{j-1})\\
          \Rightarrow &  \tau^n_j = \frac{\varepsilon^{n+1}_j - \varepsilon^n_j}{\Delta t^2} + c \frac{\varepsilon^{n}_{j + 1} - \varepsilon^{n}_{j - 1}}{2 \Delta x \Delta t} - \frac{c^2 }{2 \Delta x^2}(\varepsilon^n_{j + 1} - 2 \varepsilon^n_j + \varepsilon^n_{j-1})
        \end{align*}
        Equating the truncation error to zero, and rearranging for $\varepsilon^{n+1}_j$:
        \begin{equation*}
          \varepsilon^{n+1}_j = \varepsilon^n_j - \frac{c \Delta t}{2 \Delta x } (\varepsilon^{n}_{j + 1} - \varepsilon^{n}_{j - 1})+ \frac{c^2 \Delta t^2}{2 \Delta x^2}(\varepsilon^n_{j + 1} - 2 \varepsilon^n_j + \varepsilon^n_{j-1})
        \end{equation*}
        Which could have been seen immediately from the Lax-Wendroff method equation.
        Substituting an expansion coefficient of the Fourier series:
        \begin{equation*}
            \varepsilon (x_j, t^n) = \sum_{n = -N}^N \hat \varepsilon^n e^{ikx_j} = \sum_{n = -N}^N \hat \varepsilon^n e^{ikj \Delta x}
        \end{equation*}
        The following equation is obtained:
        \begin{align*}
                      & \hat \varepsilon^{n + 1}e^{i k j\Delta x} = \hat \varepsilon^ne^{ikj\Delta x} - \frac{c \Delta t}{2 \Delta x } (\hat\varepsilon^{n}e^{ik(j + 1)\Delta x} - \hat \varepsilon^{n}e^{ik(j - 1)\Delta x}) \\
                      & \qquad \qquad \qquad + \frac{c^2 \Delta t^2}{2 \Delta x^2}(\hat \varepsilon^ne^{ik(j + 1)\Delta x} - 2 \hat \varepsilon^ne^{ikj\Delta x} + \hat \varepsilon^ne^{ik(j-1)\Delta x}) \\
          \Rightarrow & \hat \varepsilon^{n + 1} = \hat \varepsilon^n - \frac{c \Delta t}{2 \Delta x } (\hat\varepsilon^{n}e^{ik\Delta x} - \hat \varepsilon^{n}e^{-ik\Delta x}) \\
                      & \qquad \qquad + \frac{c^2 \Delta t^2}{2 \Delta x^2}(\hat \varepsilon^ne^{ik\Delta x} - 2 \hat \varepsilon^n + \hat \varepsilon^ne^{-ik\Delta x}) \\
          \Rightarrow & \hat \varepsilon^{n + 1} = \hat \varepsilon^n\left[1  - \frac{c \Delta t}{2 \Delta x } (e^{ik\Delta x} - e^{-ik\Delta x})\right. \\
                      & \qquad \qquad \quad + \left.\frac{c^2 \Delta t^2}{2 \Delta x^2}(e^{ik\Delta x} - 2  + e^{-ik\Delta x}) \right]\\
          \Rightarrow & \hat \varepsilon^{n + 1} = \hat \varepsilon^n\left[1  - i R \sin (k \Delta x) + R^2 (\cos (k\Delta x) - 1 ) \right]\\
          \Rightarrow & S(k) = \left[1  - iR \sin (k \Delta x) + R^2 (\cos (k\Delta x) - 1 ) \right]
        \end{align*}
      \item
        Now we get:
        \begin{align*}
                      & |S(k)|^2 = \left\{\left[1 + R^2 (\cos (k\Delta x) - 1 )\right]^2 + R^2 \sin^2 (k \Delta x)  \right\}\\
          \Rightarrow & |S(k)|^2 = \left\{\left[1 - 2 R^2 \sin^2(k \Delta x / 2)\right]^2 + R^2 \sin^2 (k \Delta x)  \right\}\\
          \Rightarrow & |S(k)|^2 = \left\{1  - 4 R^2 \sin^2 (k \Delta x / 2) + 4 R^4 \sin^4 (k \Delta x / 2)+ 4R^2 \sin^2(k \Delta x / 2)\cos^2(k \Delta x / 2) \right\} \\
          \Rightarrow & |S(k)|^2 = \left\{1  - 4 R^2 \sin^2 (k \Delta x / 2)(1 - \cos^2(k \Delta x / 2)) + 4 R^4 \sin^4 (k \Delta x / 2) \right\} \\
          \Rightarrow & |S(k)|^2 = \left\{1  - 4 R^2 \sin^4 (k \Delta x / 2) + 4 R^4 \sin^4 (k \Delta x / 2) \right\} \\
          \Rightarrow & |S(k)|^2 = \left\{1  - 4 R^2 (1 - R^2) \sin^4 (k \Delta x / 2) \right\} 
        \end{align*}
        Since $R^2 \leq 1$  and $\sin^2(\theta) \leq 1$, the result $|S(k)| < 1$ follows. 
        This work demonstrates that the method is stable.
      \item
        Figure~\ref{fig:maxSk} the amplification factor's maximum value for different $R$ values.
        \begin{figure}[H]
          \centering
          \includegraphics[scale = 0.7]{../figs/maxSk.pdf}
          \caption{The Lax-Wendroff and upwind Euler method's amplification factor's maximum modulus}
          \label{fig:maxSk}
        \end{figure}
        The figure indicates that both methods are non-dissipative for $R = 1$.
        The upwind Euler method is more dissipative than the Lax-Wendroff method for $R < \frac{\sqrt{5} - 1}{2} \approx 0.618$; otherwise, the Lax-Wendroff method is more dissipative.
        \iffalse
        Finding the minimum of $S(k)$ is equivalent to maximizing:
        \begin{equation*}
          f(R, \theta) = 4 R^2 (1 - R^2) \sin^4 (\theta)
        \end{equation*}
        Since $\sin \theta \leq 1$, the maximum with respect to theta is:
        \begin{equation*}
          \max_\theta f(R, \theta) = 4 R^2 (1 - R^2)
        \end{equation*}
        Minimisation of $f$ with respect to $R$ requires:
        \begin{equation*}
          \frac{\partial f}{\partial R} = 8 R - 16 R^3 = 0 \Rightarrow R^2 = \frac{1}{2}
        \end{equation*}
        This relation implies that:
        \begin{align*}
          |S(k)|^2 \geq 1 - \frac{9}{8} + \frac{9}{16} = \frac{1}{4}
        \end{align*}
        This equation indicates nothing about the dispersivity of the Lax-Wendroff method.
        It does, however, indicate that the Lax-Wendroff method is less dissipative than the upwind-Euler scheme.
        The upwind-Euler method allows $|S(k)| = 0$, which removes contributions associated with $k$ from the numerical solution.
        \fi
        
      \item
        Figure~\ref{fig:laxWendrof} compares the Lax-Wendroff method with the exact and Euler solutions to the advection problem.
        \begin{figure}[H]
          \centering
          \includegraphics[scale = 0.92]{../figs/laxWendroff.pdf}
          \caption{Lax-Wendroff Evolution of an advection equation with $\Delta t = 0.8 \Delta x$}
          \label{fig:laxWendrof}
        \end{figure}
    \end{enumerate}
  \item
    The centered-Euler scheme for the heat equation has the error propagation:
    \begin{equation*}
      \varepsilon^{n+1}_j = \varepsilon^n_j + k \Delta t \left(\frac{\varepsilon^n_{j + 1} - 2 \varepsilon^n_j + \varepsilon^n_{j - 1}}{2 \Delta x^2 }\right) 
    \end{equation*}
    Assuming:
    \begin{equation*}
        \varepsilon (x_j, t^n) = \sum_{n = -N}^N \hat \varepsilon^n e^{ikx_j} = \sum_{n = -N}^N \hat \varepsilon^n e^{ikj \Delta x}
    \end{equation*}
    The following equation is obtained:
    \begin{align*}
                  & \hat \varepsilon^{n + 1}e^{i k j\Delta x} = \hat \varepsilon^ne^{ikj\Delta x} + \frac{k \Delta t}{\Delta x^2} \left(\hat \varepsilon^ne^{ik(j + 1 )\Delta x} - 2 \hat \varepsilon^ne^{ikj\Delta x} + \hat \varepsilon^ne^{ik(j - 1 )\Delta x}\right)\\
      \Rightarrow & \hat \varepsilon^{n + 1} = \hat \varepsilon^n + \frac{k \Delta t}{\Delta x^2} \left(\hat \varepsilon^ne^{ik\Delta x} - 2 \hat \varepsilon^n + \hat \varepsilon^ne^{-ik\Delta x}\right)\\
      \Rightarrow & \hat \varepsilon^{n + 1} = \hat \varepsilon^n \left[ 1 + \frac{k \Delta t}{\Delta x^2} \left(e^{ik\Delta x} - 2 + e^{-ik\Delta x}\right)\right]\\
      \Rightarrow & \hat \varepsilon^{n + 1} = \hat \varepsilon^n \left[ 1 + 2\frac{k \Delta t}{\Delta x^2} \left(\cos(k\Delta x) - 1\right)\right]\\
      \Rightarrow & \hat \varepsilon^{n + 1} = \hat \varepsilon^n \left[ 1 - 4\frac{k \Delta t}{\Delta x^2} \left(\sin^2(k\Delta x / 2) \right)\right]\\
      \Rightarrow & S(k) = 1 - 4\frac{k \Delta t}{\Delta x^2} \sin^2(k\Delta x / 2)
    \end{align*}
    This method is stable provided that:
    \begin{align*}
                  & |S(k)| = |1 - 4\frac{k \Delta t}{\Delta x^2} \sin^2(k\Delta x / 2)| \leq 1\\
      \Rightarrow & |1 - 4\frac{k \Delta t}{\Delta x^2} \sin^2(k\Delta x / 2)| \leq 1\\
      \Rightarrow & -2 \leq - 4\frac{k \Delta t}{\Delta x^2} |\sin^2(k\Delta x / 2)| \leq 0\\
      \Rightarrow & 0 \leq \frac{k \Delta t}{\Delta x^2} |\sin^2(k\Delta x / 2)| \leq \frac{1}{2}\\
      \Rightarrow & \frac{k \Delta t}{\Delta x^2} \leq \frac{1}{2}\\
      \Rightarrow & \Delta t \leq \frac{1}{2 k}\Delta x^2
    \end{align*}
\end{enumerate}

\end{document}
