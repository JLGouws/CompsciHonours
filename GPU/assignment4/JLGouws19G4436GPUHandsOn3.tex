\documentclass[a4paper,12pt]{article}
\usepackage{amsmath,amsfonts,amsthm,amssymb, mathtools,steinmetz, gensymb, siunitx}	% LOADS USEFUL MATH STUFF
\usepackage{xcolor,graphicx}
\usepackage[left=2cm, top=2cm, right=2cm, bottom=1cm ,a4paper]{geometry} 				% ADJUSTS PAGE
\usepackage{setspace}
\usepackage{caption}
\usepackage{tikz}
\usepackage{pgf,tikz,pgfplots}
\usepackage{mathrsfs}
\usepackage{fancyhdr}
\usepackage{float}
\usepackage{array}
\usepackage{unicode-math}
\usepackage{booktabs}
\usepackage{multirow}
\setmathfont{Libertinus Math}

\usetikzlibrary{decorations.pathreplacing,decorations.markings}
\usepgfplotslibrary{fillbetween}

\newcommand{\defeq}{\vcentcolon=}
\newcommand\block[1]{\hspace*{#1}}
\newcommand{\rpm}{\sbox0{$1$}\sbox2{$\scriptstyle\pm$}
	  \raise\dimexpr(\ht0-\ht2)/2\relax\box2 }
\pgfplotsset{compat=1.11}
	  
\newlength{\QNo}
\settowidth{\QNo}{2.}

\newlength{\QLetter}
\settowidth{\QLetter}{(a)}

\pagestyle{fancy}
\rhead{CS Honours GPUs}
\lhead{J L Gouws}

\begin{document}
\fontencoding{T1}
\fontfamily{ppl}\selectfont
\thispagestyle{empty}

{\Large \textbf{GPUs Hands On 3}} \hfill {\Large \textbf{J L Gouws}}\\
\block{1.0cm} {\large \textbf{\today}} \hfill {\large \textbf{19G4436}}\\

  \begin{table}[H]
    \resizebox{\textwidth}{!}{
      \begin{tabular}{ccccccc}
      \toprule
          \multicolumn{1}{p{4em}}{\textbf{Block Size}} & \multicolumn{1}{p{5em}}{\textbf{Kernel Execution Time(ms)}} & 
          \multicolumn{1}{p{5em}}{\textbf{Achieved occupancy(\%)}} & \multicolumn{1}{p{5em}}{\textbf{Global Memory Load Throughput(GB/S)}} & 
          \multicolumn{1}{p{5em}}{\textbf{Global Memory Load Efficiency (\%)}} &
          \multicolumn{1}{p{5em}}{\textbf{Global Memory Store Throughput (GB/S)}}&
          \multicolumn{1}{p{5em}}{\textbf{Global Memory Store Efficiency (\%)}} 
          \\ \midrule
          \midrule
        \multicolumn{7}{c}{ 2D Grid} \\
      \hline
      $64 \times  8$ & 2.66404          & 87.8    & 50.334 & 100      & 25.167& 100 \\ 
      \hline
      $2  \times 64$ & 5.88941          & 87.2    & 93     & 100      & 46.5  & 100 \\ 
      \midrule
      \midrule 
        \multicolumn{7}{c}{1D Grid}\\
        \hline
      $16 \times 16$ & 2.67481 & 89.3    & 50.028 & 100      & 25.014& 100 \\ 
      \hline
      $32 \times 32$ & 2.76861          & 83.9    & 49.929 & 100      & 24.964& 100 \\ 
      \midrule
      \midrule
        \multicolumn{7}{c}{1D Grid, 16 Data per thread, unstrided}\\
        \hline
      $2  \times 64$ & 7.04664          & 72.1    & 156.709& 12.5     & 78.354& 12.5\\ 
      \hline
      $64 \times 16$ & 7.41846          & 91.4    & 152.988& 12.5     & 76.494& 12.5\\ 
      \midrule
      \midrule
        \multicolumn{7}{c}{1D Grid, 16 Data per thread, unstrided}\\
        \hline
      $32 \times 32$ & 2.79042         & 95.2    &  48.149& 100      & 24.074& 100 \\ 
      \hline
      $16 \times 64$ & 2.83848          & 94.9    &  48.675& 100      & 24.337& 100 \\ 
      \bottomrule
    \end{tabular}
  }
  \end{table}
  Observations:\\
  The table shows that the memory throughput metrics can be misleading.
  The worst performing kernels have very high memory throughput.
  This throughput is not always productive, as shown by the memory efficiency metrics.

  Achieved occupancy can also be misleading.
  The, worst performing, unstrided 16 data per thread kernel achieves 91.4\% occupancy, yet it has an execution time of $\sim$ 7.4 ms.
  These considerations of occupancy and throughput show that one particular metric cannot give the full picture.
  Multiple metrics should be considered to understand the performance of the kernel.

  Doing sixteen consecutive data items per thread results in low memory load and store efficiencies.
  Striding solves this problem and achieves the same memory efficiency as the single datum per thread approach.
\end{document}
