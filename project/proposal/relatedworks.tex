\section{Related Works}\label{sec:relatedWorks}
In 2011 \citeauthor{Kalal2011} invented the Tracking, Learning and Detection(TLD) framework for the longterm tracking of objects in a video stream.
Kalal's original implementation uses a median flow tracker, P-N learning, and a random forrest based detector \cite{KalalPHD}.
These three components give the respective tracking, learning and detection components of the system.

The learning compenent of TLD forms the backbone of the system governing the interaction between the detector and tracker.
The three components exchange information as shown in Figure~\ref{fig:tld}, this allows the tracker to improve it's performance as time progresses \cite{Kala2011}
The system requires online learning, learning as data becomes available, in order for the system to work. 
Kalal developed the P-N Learning paradigm \cite{PNLearning}, a semi-supervised bootstrapping model \cite{murphy2012machine}, tailored to the needs of TLD.

\begin{figure}{l}{0.55\textwidth}
   \begin{tikzpicture}[approach/.style={draw,very thick, fill=white, text width=5em,
         text centered, minimum height=2em,rounded corners=3ex, scale=0.1, everynode/.style={scale=0.1}},
         idea/.style={draw=black, circle,text width=5em,
            text centered, minimum height=2.5em},
         connections/.style={->,draw=black,shorten <=2pt,shorten >=2pt},
         reverseConnections/.style={<-,draw=black,shorten <=2pt,shorten >=2pt},
      ]

      \node[draw] at (0,0) (tracking) [idea]  {Tracking};
      \node[draw] at (3.5,4.5) (learning) [idea]  {Learning};
      \draw[connections, postaction={decorate, decoration={raise=1ex, text along path, text align=center, text={|\small|Fragments of trajectory}}}] (tracking.north)to[out=90,in=190] (learning.west) ;
      \node[draw] at (7,0) (detection) [idea]  {Detection};
      \draw[connections, postaction={decorate, decoration={raise=-2.5ex, text along path, text align=center, text={|\small|Training data}}}] (learning.south) to[out=270,in=180] (detection.west) ;
      \draw[reverseConnections, postaction={decorate, decoration={raise=1ex, text along path, text align=center, text={|\small|Detections}}}] (learning.east) to[out=350,in=90] (detection.north) ;
      \draw[reverseConnections, postaction={decorate, decoration={raise=-2.5ex, text along path, text align=center, text={|\small|re-initialization}}}] (tracking.south east) to[out=-45,in=225] (detection.south west) ;
   \end{tikzpicture}
   \caption{The interaction between tracking, learning and detection in TLD. Figure from \cite{Kalal2011}}
   \label{fig:tld}
\end{figure}

\citeauthor{Enriques2014} \cite{Enriques2014} propose Kernelized Correlation Filters(KCF) and the novel Dual Correlaion filter(DCF).
Both KCF and DCF use circulant matrices and the kernel Trick.

applied on Histogram of Oriented Gradient features.

The system developed by \citeauthor{Enriques2014} does not encorporate a failure recovery mechanism--section 8 of \cite{Enriques2014}.
This is in contrast to the TLD system which provides a failure recovery mechanism in the detection component \cite{Kalal2011}.
KCFs also require a stage that will start them tracking.
KCFs have great potential to be used as the tracking stage for a TLD tracker.

Kalal uses Random Forest for detection in his implementation of TLD.
A more modern approach would be to use Extreme Gradient Boosting(XGB) for the detector.
XGB generally offers similar if not better classification performance than Random Forest and requires significantly less time to train \cite{comparativeXGB}.

There has also been investigation into the use of Convolutional Neural Networks in tracking \cite{CNNTracking}.
This offers high performance tracking of generic objects.
It will be investigated for extension to tracking and detecting people in crowds.
It will not be a main point of research unless it turns out to be fruitful--seeing as CNNs tend to require large amounts of training time, data, and memory space.

There has also been some quite old research that involves localizing multiple targets.
The research of Taylor and Drummond \cite{taylorDrummondTracking} offer this at high FPS even on low powered devices.

Currently there are vary many trackers availble all with varying degrees of performance.
These have been benchmarked in \cite{VOT2017} and \cite{VOT2020}.
This is a good reference, and will be explored further, but not all trackers can satisfy the requirements of this project. 

There has also been relevant work done on correlation tracking by \cite{Ma2015Correlation}.
Ma. et al investigated the problem of long term tracking where the target undergoes abrupt motion, heavy occlusions and disapearing from view.
There work will probably integrate well with Kernelized Correlation filters, and it has very practical advantages.
It is not the forefront of this research to handle occlusions, but it is a possible extension.

The P-N learning system is not completely unrelated to Reinforcement Learning Techniques.
There have been promising recent results in online reiforcement learning \cite{onlineRL}.
The work of \cite{onlineRL} allows for variable data budgets, which should integrate well with a constant flux of input from a video stream.
TLD may not be compatible with the methods proposed in \cite{onlineRL}, but the feaibility of the ideas will be explored in the paper.
