\section{Related Works}
\subsection{Closely related works}
In 2011 Zdenek Kalal invented the Tracking, Learning and Detection(TLD) framework for the longterm tracking of objects in a video stream\cite{Kalal2011}.
Kalal's original implementation uses a median flow tracking stage, P-N learning, and a random forrest based detector.
The system requires online(learning as data becomes available) learning in order for the system to work, Kalal developed the P-N Learning paradigm \cite{PNLearning}, a semi-supervised bootstrapping model, tailored to the needs of TLD.

There have been improvements in tracking methods since his development of TLD.
Most notably Kernelized Correlation Filters(KCF) being applied on Histogram of Oriented Gradient features\cite{Enriques2014}.
KCFs, however, do not have the ability to detect, and so if they fail they are unlikely to recover.
KCFs also require a stage that will start them tracking.
KCFs have great potential to be used as the tracking stage for a TLD tracker.

Kalal uses Random Forest for detection in his implementation of TLD.
A more modern approach would be to use Extreme Gradient Boosting(XGB) for the detector.
XGB generally offers similar if not better classification performance than Random Forest and requires significantly less time to train \cite{comparativeXGB}.


There has also been investigation into the use of Convolutional Neural Networks in tracking \cite{CNNTracking}.
This offers high performance tracking of generic objects.
It will be investigated for extension to tracking and detecting people in crowds.
It will not be a main point of research unless it turns out to be fruitful--seeing as CNNs tend to require large amounts of training time, data, and memory space.

\subsection{Less related work}
There has also been relevant work done on correlation tracking by \cite{Ma2015Correlation}.
Ma. et al investigated the problem of long term tracking where the target undergoes abrupt motion, heavy occlusions and disapearing from view.
There work will probably integrate well with Kernelized Correlation filters, and it has very practical advantages.
It is not the forefront of this research to handle occlusions, but it is a possible extension.

The P-N learning system is not completely unrelated to Reinforcement Learning Techniques.
There have been promising recent results in online reiforcement learning \cite{onlineRL}.
The work of \cite{onlineRL} allows for variable data budgets, which should integrate well with a constant flux of input from a video stream.
TLD may not be compatible with the methods proposed in \cite{onlineRL}, but the feaibility of the ideas will be explored in the paper.
