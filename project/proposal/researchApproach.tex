\section{Approach to Research}
  The first phase of this research will consist of in-depth reading of literature and further literature reviews.
  The literature reviews will start by reviewing Kalal's work on TLD.
  After a thorough review of Kalal's work, works pertaining to other trackers will be reviewed.
  Following this, there will be a further review of the literature discussing the online training of classifiers.
  The implementation of the system requires data for testing--a brief phase of data acquisition from public sources will provide data for developmental and testing purposes.
  This should suffice for the literature review and data collection.

  The second stage of this research will relate to the practicalities of the system's implementation.
  The system will be implemented in C++, this requires proficient understanding of the C++ language.
  Investigation of the openCV C++ library will be done, and the available utilities will be surveyed.

  The third phase consists of a functional reimplementation of the original TLD.
  Testing of this base system is required at this point, so that problems do not occur in later stages of the full system implementation.
  Following satisfactory performance of the TLD reimplementation, the next stage of research will commence.

  At this point further improvements to the base TLD model will be made.
  The focus of the improvement will be on the tracking and detection components of the system. 
  This involves either keeping the base model and improving the individual components or restructuring the model to improve model performance.
  This will constitute the fourth stage of the research.

  Following this, an investigation of extending the system to track multiple object simultaneously will be made.
  There are naive ways to implement a multiple object tracking system--for example creating many different single target trackers to detect and track each object.
  This stage, therefore, requires significant planning, research and reviewing the current implementation in order to acheive good model performance and efficiency.
  This will complete the implementation of the system.

  The final stage of this research involves three things.
  First, a full review of the implementation will be carried out.
  If improvements to the implementation are required and time permits, a return to stage four will be made.
  Second is a testing phase, where the complete system will be tested with videos obtained from the initial phase of the research.
  Third, a thesis will give a description of the implementation and specifications of the system.
  This completes the research project.
