\section{Approach to Research}
  Initially I will start by reading literature on tracking and facial recognition.
  This will mainly start by reviewing Kalal's work in TLD, namely his Doctoral thesis \cite{KalalPHD}.

  Next I will go into a phase of re-implementing predator in C++, with help from the OpenCV libraries.
  Once this is stable and running with decent performance, I will move onto the next stage.

  Next I will be looking at ways to improve my implementation of the TLD system.
  I will mainly focus on looking at improving the tracking and detection stages of the system.
  I will then spend a small amount of time looking at the learning component, but do not intend to completely change it.

  Next I will extend the tracking and detection system so that it is able to identify and track multiple targets in a video stream.
  This will probably require a lot of work and testing.
  It is important that the system is both reasonably accurate.
  It is particularly that the system does not get confused and misidentify objects in the start up phase.
  There is always high potential for confusion in the start up of system, where there are potentially similar looking faces.

  Finally I will be looking at ways to enhance the performance of tracking multiple objects.
  This will come later and mainly be considered and extension.
  It would be useful to have the system run on mobile devices or micro-computers, with the potential to be extended to a distributed system.
