\section{Limitations}
  It is difficult to accurately identify people from the rear.
  In this case it might be good to try and prevent the system from learning the appearance of the back of a person's head.
  A solution to this problem is also developing a system that can be distributed so as to be able to track people from a surrounded view.

  It is possible to track semi-ocluded targets, but detecting them is difficult.
  This will cause problems for the initialization and re-initialization of the tracker.
  It might be possible to impute some facial features before inputing the stream into the detection system.
  If a face is fully ocluded, however, there is not much that the system will be able to do, but a human will not be able to do very much.

  The system will require relatively good resolution cameras in order to identify people accurately from a long distance.
  In most situations involving crowds, a camera capturing a video stream will need to be placed a significant distance from the targets.
  Hence, the best we can do in software is try to use image processing techniques to artificially enlarge the target, or work on detectors that can work on minimal resolution.

  There is a problem with access to data that can be trained on, and ethics.
  Most crowd video data will be impractical to use for testing, owing to resolution and crowd size.
  There are also ethical issues to be taken into consideration.
  For testing purposes, I should be able to source my own data, but it will not necessarily resemble real world data.
