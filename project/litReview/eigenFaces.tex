\subsection{Eigenfaces}
  \citet{eigenFacesRecog} describe how to use principal component analysis (PCA) to determine features for facial recognition.
  PCA is used to determine eigenvectors, referred to as ``eigenfaces", that form a basis for the faces of concern.
  PCA can decompose any face, from a given set, into a linear combination of basis vectors; this is equivalent to representing a vector in Euclidean space in an eigenvector basis.
  A facial recognition system can use the decomposition's components to identify faces.
  
  This usage of eigenfaces allows for a more compact representation of a face.
  PCA finds a set of features that account for the most significant variation in a given collection of faces.
  This method allows \citet{eigenFacesRecog} to achieve facial recognition that is ``fast, relatively simple, and has been shown to work well in a constrained environment \cite{eigenFacesRecog}.'' 

  \citet{ICAFaceRecog} suggest using Independent Component Analysis (ICA) instead of PCA.
  ICA is a generalisation of PCA that considers the relationship of distant pixels.
  This mechanism allows ICA to encode more information in the eigenvectors compared to PCA.

  A machine learning model can use this set of features to identify a given face.
  One option is to use a neural network which offers high accuracy and quick recognition \cite{eigenFacesRecog, ICAFaceRecog}.
  Alternatively, a naive Bayes classifier which amenable to online learning \cite{murphy2012} can do the recognition.
