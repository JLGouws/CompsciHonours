\subsection{Eigenfaces}
  \citeauthor{eigenFacesRecog} \cite{eigenFacesRecog} describe how to use principle component analysis(PCA) to determine features for facial recognition.
  PCA is used to determine eigenvectors, referred to as ``eigenfaces", that form a basis for the faces of concern.
  Any face in a given set of faces can be decomposed into a linear combination of these basis vectors, as an example, this is equivalent to mathematical eigenvectors in a Euclidean space.
  The compononents of the decomposition can be used to recognize faces.
  
  This usage of eigenfaces allows for a more compact representation of a face.
  PCA finds a set of features that account for the largest amount of variation in some set of faces.
  This allowed \citeauthor{eigenFacesRecog} to acheive a method for recognizing faces that is ``fast, relatively simple, and has been shown to work well in a constrained environment \cite{eigenFacesRecog}.'' 

  \citeauthor{ICAFaceRecog} \cite{ICAFaceRecog} suggest the use of Independent Component Analysis(ICA) instead of PCA.
  ICA is a generalization of PCA that takes the relationship of distant pixels into account.
  This allows ICA to encode more information, in comparison to PCA, in the eigenvectors.

  A machine learning model can be trained on this set of features in order to identify a given face.
  One option is to use a neural network which offers high accuracy and quick recognition \cite{eigenFacesRecog, ICAFaceRecog}.
  Another option is to use a naive bayes classifier which is amenable to online learning \cite{murphy2012}.
