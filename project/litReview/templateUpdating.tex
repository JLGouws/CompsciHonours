\subsection{Updating Template Trackers}
  Template tracking \cite{templateUpdate} assumes that the appearace of the target object does not undergo changes.
  The results in simplistic tracking and the tracker will fail if the target undergoes a change in orientation or a change of view.
  \citeauthor{templateUpdate} propose solutions to this, and discuss the problems around the solutions.
  The problems are a result of what is known as the stability-plasticity dilema \cite{grossberg1987}.

  Everytime a tracker template is updated, some error in the template is introduced.
  This causes the tracker to drift, and evetually cause the tracker to fail \cite{templateUpdate}.

  Suppose that $\mathbf{x}$ is the coordinate vector of a pixel in the $n^{\text{th}}$ frame $I_n(\mathbf{x})$ of a video.
  Let $T(\mathbf{x})$ be the template of the target image, and $T_n(\mathbf{x})$ be the template of the object in the $n^\text{th}$ frame of a video sequence.
  The warp of the image $\mathbf{W}(\mathbf{x};\mathbf{p})$ represents the allowed deformations of the template given a set of parameters $\mathbf{p}$ which define a deformation.
  The warp maps a pixel from the template frame to the coordinates of the video frame $I_n(\mathbf{x})$.

  Given these defintions, the problem of tracking formally reduces to computing the parameters for the deformation of the object:
  \begin{equation}
    \mathbf{p}_n = \arg \min_\mathbf{p} \sum_{\mathbf{x} \in T_n}\left[I_n(\mathbf{W}(\mathbf{x};\mathbf{p})) - T_n(\mathbf{x})\right]^2
    \label{eq:newTempParams}
  \end{equation}
  And then updating the tracking template based on the warp of the $n^\text{th}$ frame, for example a naive update is \cite{templateUpdate}:
  \begin{equation*}
    \forall n \geq 1, T_{n+1}(\mathbf{x}) = I_n(\mathbf{W}(\mathbf{x};\mathbf{p}_n))
  \end{equation*}

  Implementing this requires a gradient descent algorithm for non-linear optimizations.
  Equation \ref{eq:newTempParams} now becomes:
  \begin{equation}
    \mathbf{p}_n^* = \mathrm{gd} \min_{\mathbf{p} = \mathbf{p}_{n-1}} \sum_{\mathbf{x} \in T_n}\left[I_n(\mathbf{W}(\mathbf{x};\mathbf{p})) - T_n(\mathbf{x})\right]^2
    \label{eq:paramUpdate}
  \end{equation}
  With $\mathrm{gd} \min_{\mathbf{p}}$ indicating a gradient descent minimization starting from the warp parameters of the $(n-1)^\text{th}$ frame.
  
  Using Equation~\ref{eq:paramUpdate}, \citeauthor{templateUpdate} suggest a template update with drift correction given by:
  \begin{align*}
    &\text{If } \norm{\mathbf{p_n}^* - \mathbf{p_n}} \leq \varepsilon \text{ Then } T_{n+1} (\mathbf{x}) = I_n(\mathbf{W}(\mathbf{x};\mathbf{p}_n^*))\\
    &\text{else } T_{n+1}(\mathbf{x}) = T_n(\mathbf{x})
  \end{align*}
  Given some small threshold $\varepsilon > 0$.
  This updates the template if retaining the template would cause tracker drift, otherwise the template is not updated.
