\section{Related Works}\label{sec:relatedWorks}
In 2011 \citeauthor{Kalal2011} invented the Tracking, Learning and Detection(TLD) framework for the longterm tracking of objects in a video stream.
Kalal's original implementation uses a median flow tracker, P-N learning, and a random forrest and nearest neighbour based detector \cite{KalalPHD}.
These three components give the respective tracking, learning and detection components of the system.

The learning compenent of TLD forms the backbone of the system, governing the interaction between the detector and tracker.
The three components exchange information as shown in Figure~\ref{fig:tld}, this allows the tracker to improve it's performance as time progresses \cite{Kalal2011}.
For the system to operate, it requires online learning--learning as data becomes available. 
Kalal developed the P-N Learning paradigm \cite{PNLearning}, a semi-supervised bootstrapping model \cite{murphy2012}, tailored to the needs of TLD.

\begin{figure}[!ht]
  \centering
   \begin{tikzpicture}[approach/.style={draw,very thick, fill=white, text width=5em,
         text centered, minimum height=2em,rounded corners=3ex, scale=0.1, everynode/.style={scale=0.1}},
         idea/.style={draw=black, circle,text width=5em,
            text centered, minimum height=2.5em},
         connections/.style={->,draw=black,shorten <=2pt,shorten >=2pt},
         reverseConnections/.style={<-,draw=black,shorten <=2pt,shorten >=2pt},
      ]

      \node[draw] at (0,0) (tracking) [idea]  {Tracking};
      \node[draw] at (3.5,4.5) (learning) [idea]  {Learning};
      \draw[connections, postaction={decorate, decoration={raise=1ex, text along path, text align=center, text={|\small|Fragments of trajectory}}}] (tracking.north)to[out=90,in=190] (learning.west) ;
      \node[draw] at (7,0) (detection) [idea]  {Detection};
      \draw[connections, postaction={decorate, decoration={raise=-2.5ex, text along path, text align=center, text={|\small|Training data}}}] (learning.south) to[out=270,in=180] (detection.west) ;
      \draw[reverseConnections, postaction={decorate, decoration={raise=1ex, text along path, text align=center, text={|\small|Detections}}}] (learning.east) to[out=350,in=90] (detection.north) ;
      \draw[reverseConnections, postaction={decorate, decoration={raise=-2.5ex, text along path, text align=center, text={|\small|re-initialization}}}] (tracking.south east) to[out=-45,in=225] (detection.south west) ;
   \end{tikzpicture}
   \caption{The interaction between tracking, learning and detection in TLD. Figure from \cite{Kalal2011}}
   \label{fig:tld}
\end{figure}

\citeauthor{Enriques2014} \cite{Enriques2014} propose Kernelized Correlation Filters(KCF) and the novel Dual Correlaion filter(DCF).
Both KCF and DCF use circulant matrices and the kernel Trick.
The implementation of KCF by \citeauthor{Enriques2014} uses a Gaussian Kernel, whereas the DCF implementation uses a linear kernel.
The calculations involved with the linear kernel are less computationally complex than KCF. 
DCF can, hence, be processed faster, but, at the cost of some tracking precision.

Work by \citeauthor{multichannelCorrFilters} \cite{multichannelCorrFilters} allows KCF and DCF to be applied to modern and useful feature descriptors.
\citeauthor{Enriques2014} show that KCF and DCF be be applied to Histogram of Oriented Gradient(HOG) features to track and detect objects in a video stream with lower computation times and better accuracy.
KCF and DCF applied to HOG features are shown to outperfrom many tracking systems Table~\ref{tab:trackers}.
The results shown by Table~\ref{tab:trackers} are obtained from running the algorithms on a standard four core desktop processor from \citedate{Enriques2014}.

\begin{table}
  \centering
  \begin{tabular}[t]{cccc}
    \toprule
    Algorithm & feature & Mean precision & Mean FPS \\
    \midrule
    KCF       & HOG     & 73.2\%         & 172      \\
    \hline
    DCF       & HOG     & 72.8\%         & 292      \\
    \hline
    KCF       & Raw pixels & 56.0\%      & 154      \\
    \hline
    DCF       & Raw pixels & 45.1\%      & 278      \\
    \midrule
    \midrule
    \multicolumn{2}{c}{TLD}   & 60.8\%      &  28      \\
    \hline
    \multicolumn{2}{c}{Struck\cite{struck}}& 65.6\%     &  20     \\
    \hline
    \multicolumn{2}{c}{MOSSE\cite{mosse}}& 43.1\%      &  615     \\
    \bottomrule
  \end{tabular}
  \caption{Comparison of various trackers, adapted from \cite{Enriques2014}}
  \label{tab:trackers}
\end{table}

The system implemented by \citeauthor{Enriques2014} does not, however, encorporate a failure recovery mechanism--section 8 of \cite{Enriques2014}.
In other words \citeauthor{Enriques2014} only explore KCF in the domain of ST.
This is in contrast to the original TLD system which provides a failure recovery mechanism in the detection component \cite{Kalal2011}.
The ST using KCF and DCF done by \citeauthor{Enriques2014} can be used in a TLD framework for LT.

\citeauthor{Ma2015Correlation} \cite{Ma2015Correlation} investigate the problem of single object LT using correlation tracking.
\citeauthor{Ma2015Correlation} use two Gaussian ridge regression \cite{murphy2012} models for tracking.
One model uses the relative change in background and target as time progresses, the other model tracks by using the target's appearance.
The first model is used to track the object's trajectory through fast motion and occlusions, and the second is used for scale change.
Using both tracking models they train an online detector that is both flexible(from first tracker model) and stable(from second tracker model).

\citeauthor{Ma2015Correlation} train a random fern classifier \cite{ferns2007} \cite{Kalal2011} online in order to handle tracker failure.
This solves the LT problem in a similar way to \citeauthor{KalalPHD}.

The Visual Object Tracking Challenge(VOT) is a challenge that benchmarks various trackers every year \cite{VOT2017} \cite{VOT2020}.
VOT investigates both ST and LT.
In recent years, VOT has also introduced a real time challenge \cite{VOT2020}.

There has also been investigation into the use of Convolutional Neural Networks in tracking \cite{CNNTracking}.
The work by \citeauthor{CNNTracking} offers high performance tracking of generic objects.
The system implemented by \citeauthor{CNNTracking} requires a large amount of offline training.

\citeauthor{onlineRL} \cite{onlineRL} propose a new online reinforcement learning method that can be used to train models with minimal input data.
\citeauthor{onlineRL} describe the \textit{Reanalyse} algorithm.
Given a state of a machine learning model the \textit{Reanalyse} algorithm generates training targets for the model from some input data.
When the model has improved by training, the \textit{Reanalyse} algorithm generates more training targets based on the new state of the model, the already seen input data, and any new input data.
The algorithm allows the available training data to be cycled--this allows the algorithm to extract most of the information from a limited dataset.
