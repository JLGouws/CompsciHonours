\subsection{Tracking with Correlation Filters}
\citeauthor{Enriques2014} \cite{Enriques2014} propose Kernelized Correlation Filters(KCF) and the novel Dual Correlaion filter(DCF).
Both KCF and DCF use circulant matrices and the kernel Trick.
The implementation of KCF by \citeauthor{Enriques2014} uses a Gaussian Kernel, whereas the DCF implementation uses a linear kernel.
The calculations involved with the linear kernel are less computationally complex than KCF. 
DCF can, hence, be processed faster, but, at the cost of some tracking precision.

Work by \citeauthor{multichannelCorrFilters} \cite{multichannelCorrFilters} allows KCF and DCF to be applied to modern and useful feature descriptors.
\citeauthor{Enriques2014} show that KCF and DCF be be applied to Histogram of Oriented Gradient(HOG) features to track and detect objects in a video stream with lower computation times and better accuracy.
KCF and DCF applied to HOG features are shown to outperfrom many tracking systems Table~\ref{tab:trackers}.
The results shown by Table~\ref{tab:trackers} are obtained from running the algorithms on a standard four core desktop processor from \citedate{Enriques2014}.

\begin{table}
  \centering
  \begin{tabular}[t]{cccc}
    \toprule
    Algorithm & feature & Mean precision & Mean FPS \\
    \midrule
    KCF       & HOG     & 73.2\%         & 172      \\
    \hline
    DCF       & HOG     & 72.8\%         & 292      \\
    \hline
    KCF       & Raw pixels & 56.0\%      & 154      \\
    \hline
    DCF       & Raw pixels & 45.1\%      & 278      \\
    \midrule
    \midrule
    \multicolumn{2}{c}{TLD}   & 60.8\%      &  28      \\
    \hline
    \multicolumn{2}{c}{Struck\cite{struck}}& 65.6\%     &  20     \\
    \hline
    \multicolumn{2}{c}{MOSSE\cite{mosse}}& 43.1\%      &  615     \\
    \bottomrule
  \end{tabular}
  \caption{Comparison of various trackers, adapted from \cite{Enriques2014}}
  \label{tab:trackers}
\end{table}

The system implemented by \citeauthor{Enriques2014} does not, however, encorporate a failure recovery mechanism--section 8 of \cite{Enriques2014}.
In other words \citeauthor{Enriques2014} only explore KCF in the domain of ST.
This is in contrast to the original TLD system which provides a failure recovery mechanism in the detection component \cite{Kalal2011}.
The ST using KCF and DCF done by \citeauthor{Enriques2014} can be used in a TLD framework for LT.

\citeauthor{Ma2015Correlation} \cite{Ma2015Correlation} investigate the problem of single object LT using correlation tracking.
\citeauthor{Ma2015Correlation} use two Gaussian ridge regression \cite{murphy2012} models for tracking.
One model uses the relative change in background and target as time progresses, the other model tracks by using the target's appearance.
The first model is used to track the object's trajectory through fast motion and occlusions, and the second is used for scale change.
Using both tracking models they train an online detector that is both flexible(from first tracker model) and stable(from second tracker model).

\citeauthor{Ma2015Correlation} train a random fern classifier \cite{ferns2007, Kalal2011} online in order to handle tracker failure.
This solves the LT problem in a similar way to \citeauthor{KalalPHD} \cite{KalalPHD}.
