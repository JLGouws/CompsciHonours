\subsection{Approach to Research}
  The first phase of this research will consist of an in-depth reading of literature and further literature reviews.
  The literature reviews will start by reviewing Kalal's work on TLD.
  After a thorough review of Kalal's work, the research requires a review of works describing trackers other than TLD.
  Following this, there will be a further review of the literature discussing the online training of classifiers.
  Implementing the system requires data for testing--a brief phase of data acquisition from public sources will provide data for developmental and testing purposes.
  This material accumulation should suffice for the literature review and data collection.

  The second stage of this research will relate to the practicalities of the system's implementation.
  The language of the system's development is C++; The implementation requires a proficient understanding of the C++ language.
  Implementation requires an Investigation of the OpenCV C++ library and other available utilities at this point of the approach.

  The third phase consists of a functional reimplementation of the original TLD.
  Testing of this base system is required at this point so that problems do not occur in later stages of the full system implementation.
  Following the satisfactory performance of the TLD reimplementation, the next stage of research will commence.

  At this point, the base TLD system requires further improvement.
  The improvement will focus on the tracking and detection components of the system. 
  These improvements involve keeping the base model and improving the individual tracker parts or restructuring the model to improve performance.
  This phase will constitute the fourth stage of the research.

  Following this, the research investigates extending the system to track multiple objects simultaneously.
  There are naive ways to implement many object tracking systems, for example, creating many different single-target trackers to detect and track each object. 
  This stage requires significant planning, research and reviewing of the current implementation to achieve good model performance and efficiency.
  This stage completes the implementation of the system.

  The final stage of this research involves three things.
  First, the system requires a full review to find the strengths and faults of the design.
  If improvements to the implementation are required and time permits, development returns to stage four.
  Second, the system does test runs on videos obtained from the initial phase of the research for performance evaluation.
  Third, a thesis describes the implementation and specifications of the system.
  This step completes the research project.
