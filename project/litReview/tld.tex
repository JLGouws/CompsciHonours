\subsection{Tracking, Learning, Detection}\label{sec:tld}

  \begin{figure}[!ht]
    \centering
     \begin{tikzpicture}[approach/.style={draw,very thick, fill=white, text width=5em,
           text centered, minimum height=2em,rounded corners=3ex},
           idea/.style={draw=black, circle,text width=5em,
              text centered, minimum height=2.5em},
           connections/.style={->,draw=black,shorten <=2pt,shorten >=2pt},
           reverseConnections/.style={<-,draw=black,shorten <=2pt,shorten >=2pt},
           scale=0.96, everynode/.style={scale = 0.96},
        ]

        \node[draw] at (0,0) (tracking) [idea]  {Tracking};
        \node[draw] at (3.5,4.5) (learning) [idea]  {Learning};
        \draw[connections, postaction={decorate, decoration={raise=1ex, text along path, text align=center, text={|\small|Fragments of trajectory}}}] (tracking.north)to[out=90,in=190] (learning.west) ;
        \node[draw] at (7,0) (detection) [idea]  {Detection};
        \draw[connections, postaction={decorate, decoration={raise=-2.5ex, text along path, text align=center, text={|\small|Training data}}}] (learning.south) to[out=270,in=180] (detection.west) ;
        \draw[reverseConnections, postaction={decorate, decoration={raise=1ex, text along path, text align=center, text={|\small|Detections}}}] (learning.east) to[out=350,in=90] (detection.north) ;
        \draw[reverseConnections, postaction={decorate, decoration={raise=-2.5ex, text along path, text align=center, text={|\small|re-initialization}}}] (tracking.south east) to[out=-45,in=225] (detection.south west) ;
     \end{tikzpicture}
     \caption{The interaction between tracking, learning and detection in TLD. Figure from \cite{Kalal2011}}
     \label{fig:tld}
  \end{figure}

  In 2011 \citeauthor{Kalal2011} invented the Tracking, Learning and Detection(TLD) framework for the long-term tracking of objects in a video stream.
  Kalal's original implementation uses a median flow tracker, P-N learning, and a random forest and nearest neighbour based detector \cite{KalalPHD}.
  These three components give the respective tracking, learning and detection components of the system.

  The learning component of TLD forms the backbone of the system, governing the interaction between the detector and tracker.
  The three components exchange information as shown in Figure~\ref{fig:tld}, this allows the tracker to improve its performance as time progresses \cite{Kalal2011}.
  By the nature of TLD, online learning is required for the learning component.
  Kalal developed the P-N Learning paradigm \cite{PNLearning}, a semi-supervised bootstrapping model \cite{murphy2012}, tailored to the needs of TLD.

  The tracker of TLD outputs a bounding box for the target object in every frame.
  A second bounding box for the target object is also produced by the detector.
  The P-experts and N-experts of the learning component use these bounding boxes to determine the false positives and the false negatives of the detector.
  This is used to update the detector, as described in Section \ref{sec:pnlearning}.
  An integrator is then used to combine the bounding boxes given by the tracker and detector \cite{Kalal2011}.
