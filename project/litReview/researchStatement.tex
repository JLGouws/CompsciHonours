\section{Reseach Statement}
  This section revises of the problem statement stated in the project proposal.
  The problem statement has been restated for clarity, but the content of the statement is the same.
  \begin{itemize}
    \item
      The design and implementation of a long-term tracking system--given minimal input data, the system can count the number and measure the duration of appearances of multiple target faces in a single video stream--can use machine learning and computer vision techniques.
  \end{itemize}

  The central computer vision technique that is tested is the Tracking, Learning, and Detection framework--discussed in Section~\ref{sec:tld}.
  This framework allows for long-term tracking with minimal input data. 
  
  Long-term tracking(LT) is tracking of objects that can undergo partial oclusions, change appearance, and disappear and reappear from the field of view.
  This is opposed to short-term tracking(ST) where the object remains fully in the field of view for the whole duration of tracking.
  ST can be used as a basis for LT, as is done in the case of TLD.

  The definition of minimal input data, in the context of this research, is a single image for each face that is required to be tracked, where each image includes at least one bounding box.
  The bounding box defines the location of the face in the image and a label for the face.
  The goal of this research is to implement a working system that meets the conditions specified by the Research Statement.
