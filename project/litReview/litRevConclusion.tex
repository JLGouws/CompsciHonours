  \subsection{Conclusion of the Literature Review}
    The Visual Object Tracking Challenge(VOT) ranks many current tracking systems every year \cite{VOT2017, Kristan2020a}.
    The challenge ranks different categories of trackers--the two main categories are long-term and short-term tracking.
    VOT ranks the trackers based on various metrics that indicate the tracker's accuracy.

    The VOT 2017 challenge introduced a real-time tracking challenge, which uses the short-term tracking dataset and performance measures \cite{Kristan2020a}.
    VOT does not, however, require long-term trackers to run at real-time frame rates; some trackers run at less than one frame a second.
    Thus, many of the tracking systems ranked in VOT are not amenable to this research.

    There are, at present, two main approaches to solving the problem of long-term tracking in real-time.
    The first approach uses correlation tracking and training a classifier online \cite{Ma2015Correlation, Enriques2014, Kalal2011}.
    These methods require little set-up time and can track single unknown objects in a video stream.
    
    Most modern correlation trackers, seen in VOT, use deep features and neural networks.
    Correlation trackers dominate the VOT short term challenge and are also feature in the long-term challenge.

    The second approach uses CNN's \cite{bertinetto2016}.
    This approach requires an offline training stage, but the offline training stage need not be repeated.
    CNN's allow for high accuracy tracking \cite{CNNTracking, bertinetto2016}, and provided that online learning is not required, the trackers can process videos at high frame rates \cite{Kristan2020a}.

    From a user's point of view, the operation of the CNN approach and the correlation approach seem identical.
    After the CNN has undergone offline learning, the CNN tracking systems require a single image with a bounding box to track an unknown object \cite{CNNTracking, bertinetto2016}.
