\subsection{Multi-Pose Face Recognition}
  In a realistic video stream, it is atypical for all the faces to be facing the camera at a given time.
  The faces might change pose from frontal to profile.
  Thus, for application to real-world data, facial recognition must identify faces from a frontal and portrait pose.

  \citet{viewBasedFaceRecog} suggest two solutions to the problem of Multi-Pose Face Recognition using eigenfaces.
  The first solution is using a single high dimensional eigenface space.
  This face space uses the basis vectors to represent information about both face and pose.
  Alternatively, different spaces can represent different viewpoints--PCA defines each face space by using all the images displaying a specific face angle.
  One face space uses photographs taken 10 degrees left of the frontal view; Another face space uses images of faces at an angle of 20 degrees right of the frontal view.

  \citet{PoseFaceRec} describe ways to recognise and track a face that takes on multiple poses.
  The system proposed by \citet{PoseFaceRec} has three components: Haar Cascades based face detection, weighted modular PCA based face recognition and Kalman tracker.
