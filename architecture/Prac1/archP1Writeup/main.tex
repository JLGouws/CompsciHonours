\documentclass[12pt,a4]{article}
\usepackage{physics, amsmath,amsfonts,amsthm,amssymb, mathtools,steinmetz, gensymb, siunitx}	% LOADS USEFUL MATH STUFF
\usepackage{xcolor,graphicx}
\usepackage[left=45pt, top=60pt, right=45pt, bottom=65pt ,a4paper]{geometry} 				% ADJUSTS PAGE
\usepackage{setspace}
\usepackage{caption}
\usepackage{tikz}
\usepackage{pgf,tikz,pgfplots,wrapfig}
\usepackage{mathrsfs}
\usepackage{fancyhdr}
\usepackage{float}
\usepackage{array}
\usepackage{booktabs,multirow}
\usepackage{bm}
\usepackage{fancyvrb}


\usetikzlibrary{decorations.text, calc}
\pgfplotsset{compat=1.7}

\usetikzlibrary{decorations.pathreplacing,decorations.markings}
\usepgfplotslibrary{fillbetween}

\newcommand{\vect}[1]{\boldsymbol{#1}}

\usepackage{hyperref}
%\usepackage[style= ACM-Reference-Format, maxbibnames=6, minnames=1,maxnames = 1]{biblatex}
%\addbibresource{references.bib}


\AtBeginDocument{\hypersetup{pdfborder={0 0 0}}}

\title{
\textsc{Architecture Prac 1}
}
\author{\textsc{J L Gouws}
}
\date{\today
\\[1cm]}



\usepackage{graphicx}
\usepackage{array}

\VerbatimFootnotes


\begin{document}
\thispagestyle{empty}

\maketitle

\begin{enumerate}
  \item
    \verb|handleReference| is the entry point into handling the multiple level associative cache.
    It coordinates the handling of a reference line that appears in the trace.

    First, \verb|handleReference| searches the cache for for the reference.
    The call to \verb|findInCache| handles the search in the cache for the reference.
    This search obtains the highest level of cache that contains required datum.

    Second, \verb|handleReference| determines if the data is available to the CPU(that is in L1 cache), or not.
    If the data is available in L1 cache, the reference has been handled from the perspective of cache.
    At this point \verb|handleReference| can update the appropriate statistics (number of hits and total time cost of hits) for L1 cache hits.

    Otherwise, the data item was not found in L1 cache, and a miss to L1 is handled.
    L1 cache's \verb|misscost| is increased with the L1 access time.\footnote{
      The code starts to look slightly weird at this point. 
      Line \verb|311| defines a variable named \verb|misscost| that is never used.
      This variable is set to the \verb|hittime| of the cache layer that contains the desired datum/instruction.
      The function \verb|findInCache| uses this cost to update L1's \verb|misscost|, so it should not be used again in \verb|handleReference|.
    }
    This scheme appears to have the intention of modelling a write to L1 cache.
    The L1 cache will take an access cycle to be written to.
    Then \verb|handleMiss| handles a miss through the other layers of cache.

    There appears to be a bug in the way that the statistics are updated.
    The function \verb|findInCache| updates the \verb|misscost| of L1 cache with \verb|hittime| of the cache where the datum was found.
    The function \verb|handleMiss| updates the \verb|misscost| of L1 cache with the next lower level cache's \verb|hittime|.
    If there is a miss in L1 cache and the datum/instruction is found in L2, L1's \verb|misscost| will be penalized by L2's \verb|hittime| twice.
    I doubt this modelling is reallistic--I do not believe that L2 cache would have to be accessed twice in this case.

    See also line \verb|240| of \verb|multilevelAssoc.c|, the function \verb|findInCache| updates L1D's \verb|misscost|, but it should update L1I's \verb|misscost|.

    Another spurrious error occurs in \verb|handleMiss|.
    The function \verb|handleMiss| updates \verb|i|-th layer's \verb|misscost| by the \verb|i + 1|-th layer's \verb|hitcost|, see line \verb|388| of \verb|multilevelAssoc.c|.
    If L1 cache is split then L1I's \verb|misscost| increases by L1D's \verb|hitcost|.
    Obviously, a realy instruction will not move from L1D to L1I, and L1I's \verb|misscost| must increase by L2's \verb|hitcost|.

  \item
\end{enumerate}

\end{document}
